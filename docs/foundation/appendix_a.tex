\section{\\ Palm Probability}\label{app:AppendixA}
% the \\ insures the section title is centered below the phrase: AppendixA
%
Palm characteristics are probabilities or means that refer to individual points in a PP.  Meaning that we want statistics from individual points perspectives.  The usual approach considers a point at the origin \textbf{0}. However, the probability that a stationary point process has a point exactly at \textbf{0} is zero. Therefore, the probability that a PP $\Phi$ has some property provided that it has a point at \textbf{0} is a difficult quantity.
\par
%
We determine the mean and probability related to an event of having a point at \textbf{0} as follows.  Consider an observation window W in which $\Phi$ has $\Phi(W) = n$ points. These points $x_1,..,x_n$ are taken in turn and $\Phi$ is shifted such that the relevant point lies at the origin \textbf{0}.  This process is repeated for each $x_i$.  For each shift, the number of points is counted within $r$ i.e. $b(\textbf{0},r)$.  Their average yields an estimate of the mean number of points in a sphere of radius $r$ centered at a PP point, where in all cases the point $x_i$ itself is never counted.  Second, each point is then marked.  If within the radius $r$, it will receive a mark $1$ and $0$ otherwise.  Then all the marks are then added and divided by $n$, the number of points in $\Phi$.  This is an estimate of the probability that a point in the PP has its nearest neighbor at a distance less than $r$~\cite{Illian2008}.
\par
%
Mathematically in the Palm sense, the mean and probabilities of having a point within $r$ are show in equations~\eqref{eq:mean} and \eqref{eq:prob} respectively.  Where the index $0$ in $\textbf{E}_0$ and $P_0$ indicates the shifting of the patterns towards $\textbf{0}$ in the spatial plain.
%
\begin{equation}\label{eq:mean}
  \textbf{E}_0(\Phi(b(\textbf{0},r) \setminus \{\textbf{0}\}))
\end{equation}
\begin{equation}\label{eq:prob}
  P_0(\Phi(b(\textbf{0},r) \setminus \{\textbf{0}\})>0)
\end{equation}
%
To provide even futher relation to more common probability tools the Palm probability $P_{0}$ can be written in the following way:
%
\begin{equation}
  P_{0}(\Phi \in \mathit{A}) = \frac{\textbf{E}\Big(\sum_{x \in \Phi \cap W } (\mathbb{1}_{\mathit{A}} (\Phi - x))\Big)}{\lambda v(W)}
\end{equation}
%
Here $W$ is some test set of positive volume $v(W)$, and $\Phi \in \mathit{A}$ is a general notation for point process $\Phi$ has property $\mathit{A}$. Clearly, $\mathit{A}$ has to be a property
which makes sense for a point process with a point at $\textbf{0}$. An example of this is $\Phi(b(\textbf{0},r) \setminus \{\textbf{0}\})=0$. The indicator $\mathbb{1}_{\mathit{A}} (\Phi - x)$ is $1$ if the shifted point process $\Phi - x$ has property $\mathit{A}$ and $0$ otherwise.  Mecke (1967) showed that $P_0$ is independent of choice of $W$.
