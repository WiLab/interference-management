\section{LTE Interference}
%
In this section we will discuss the application of LTE femtocells (FC) in the stochastic geometry framework.  We will only be considering co-channel interference among FC, assuming they are operating in disjoint channels from other LTE base stations such as Macrocells (MC).  FC's are assumed to be uncoordinated nodes that are uniformly spread across the environment.  Such nodes are uncoordinated because they lack the low-latency dedicated links that exists at the higher tiered nodes, known as X2 links.  MC
%
\todo[size=\small, color=green!40]{Add more background about LTE and femtocells}%
\par
%
FC's utilize an orthogonal frequency division multiple access (OFDMA) medium access (MAC) scheme to transmit data to mobile users called UE's in the downlink.
\todo[size=\small, color=green!40]{Add info about resource block terminology}%
\par
%
We extend the notions from the slotted Aloha based analysis into this framework.  First we will provide a simplified model assuming a fixed resource usage or fractional usage of resources.  Then we will add additional complexity by providing a stochastic traffic model to the resource allocation of each FC.  This traffic model will be discussed first, and derivations provided into the geometry framework.
\par
%
At a high level if we ignore UE feedback, we can model resource selection as an independent thinning of the network intensity $\lambda_{f}$ by the fraction of channels used $p$, creating a new PPP with intensity $\lambda_{tf}$.  
