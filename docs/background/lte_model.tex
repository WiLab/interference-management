\section{LTE Interference}
%
In this section we will discuss the application of LTE femtocells (FC) in the stochastic geometry framework.  We will only be considering co-channel interference among FC, assuming they are operating in disjoint channels from other LTE base stations such as Macrocells (MC).  FC's are assumed to be uncoordinated nodes that are uniformly spread across the environment.  Such nodes are uncoordinated because they lack the low-latency dedicated links that exists at the higher tiered nodes, known as X2 links.  MC
%
\todo[size=\small, color=green!40]{Add more background about LTE and femtocells}%
\par
%
FC's utilize an orthogonal frequency division multiple access (OFDMA) medium access (MAC) scheme to transmit data to mobile users called UE's in the downlink.
\todo[size=\small, color=green!40]{Add info about resource block terminology}%
\par
%
We extend the notions from the slotted Aloha based analysis into this framework.  First we will provide a simplified model assuming a fixed resource usage or fractional usage of resources.  Then we will add additional complexity by providing a stochastic traffic model to the resource allocation of each FC.  This traffic model will be discussed first, and derivations provided into the geometry framework.
\par
%
At a high level if we ignore UE feedback\footnote{This is reasonable because...}, we can model resource selection as an independent thinning of the network intensity $\lambda_{f}$ by the fraction of channels used $p$, creating a new PPP with intensity $\lambda_{tf}$.  This thinning reduces the interference seen at each network node.  To understand how this effects performance we must define performance metrics and their relations to interference.\par
%
Assuming constant power $P$ of each network node and noise power $N$, the SINR at each receiver is shown in equation~\eqref{eq:SINR}.
%
\begin{equation}\label{eq:SINR}
  SINR(\textbf{x}_i) = \frac{S}{I(\textbf{x}_i) + N}
\end{equation}
%
$S$ and $I$, the power received from the desired transmitted and combined interferers, are provided in equation~\eqref{eq:Powers}.  $I$ is a shot noise (SN) process, formed by the sum of all spatially random points of $\Phi$.  To remove the path loss singularity around the receiver itself we will assume a minimum distance $\epsilon$ between the receiver and all transmitters.  The channel $h$ is a positive random variable with unit mean.
%
\begin{equation}\label{eq:Powers}
  S = Pl_{\alpha,\epsilon}(u) = Ph_iu^{-\alpha},\quad I(\textbf{x}_i)=\sum_{k\in \Phi,k\neq i} Ph_kl_{\alpha,\epsilon}(|x_i-x_k|)
\end{equation}
%
From SINR we can develop the performance metrics of the network.  We will begin with outage probability $q$.  Outage probability is defined in equation~\eqref{eq:outage} as the probability of SINR values below some threshold $T$.
%
\begin{equation}\label{eq:outage}
  q(\lambda) = P{SINR<T}
\end{equation}
%
\section{Modifications}
%
Vary number of resources used by FC across different intensities.\par
%
Have a stochastic resource usages, defined by a traffic model.\par
%
