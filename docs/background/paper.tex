\documentclass[a4paper]{article}

\usepackage[english]{babel}
\usepackage[utf8]{inputenc}
\usepackage{amsmath}
\usepackage{amssymb}
\usepackage{graphicx}
\usepackage[colorinlistoftodos]{todonotes}

\title{Interference Modeling in Small Cell Networks}

\author{Travis F. Collins and Srikanth Pagadarai}

\date{\today}

\begin{document}
\maketitle

\section{Mathematical Background}
The purpose of this document is to provide the mathematical foundation to understand analytical network capacity analysis under spatial structures using point processes (PP).
The network we are modeling contains an infinite number of transmitters denoted by the homogeneous Poisson PP (PPP) $\Phi$, with intensity measure $\lambda$.  A point process is simply a collection of random variables spatial distributed on a plain according to specified distribution.  A PPP can be informally understood as a PP with uniform spaces points in an area \textit{B} with the number of points in areas of size \textit{B} being Poisson distributed with mean $\lambda$. For simplicity we will only be working with two dimensional plains, points $x\in\Phi$ and $x\in\mathbb{R}^2$, avoiding much of the confusing introduced by heavy measure theory based work.  $\lambda$ is a first order statistical property of a PP, defined as the mean number of points/events  $\textit{N}$ per unit area/volume $\textit{v}$ in any set $\textit{B}$~\cite{Illian2008}.  Equation~\eqref{eq:intensity} show this relation, where $\textbf{E}$ represents the expectation operation.  This is analogous to density of points in region or area.\par
%
\begin{equation}\label{eq:intensity}
\lambda = \frac{\textbf{E}[N(B)]}{v(B)}
\end{equation}
%
\subsection{Palm Probability}
%
Palm characteristics are probabilities or means that refer to individual points in a PP.  Meaning that we want statistics from individual points perspectives.  The usual approach considers a point at the origin \textbf{0}. However, the probability that a stationary point process has a point exactly at \textbf{0} is zero. Therefore, the probability that a PP $\Phi$ has some property provided that it has a point at \textbf{0} is a difficult quantity.\par
%
We determine the mean and probability related to an event of having a point at \textbf{0} as follows.  Consider an observation window W in which
$\Phi$ has $\Phi(W) = n$ points. These points $x_1,..,x_n$ are taken in turn and $\Phi$ is shifted such that the relevant point lies at the origin \textbf{0}.  This process is repeated for each $x_i$.  For each shift, the number of points is counted within $r$ i.e. $b(\textbf{0},r)$.  Their average yields an estimate of the mean number of points in a sphere of radius $r$ centered at a PP point, where in all cases the point $x_i$ itself is never counted.  Second, each point is then marked.  If within the radius $r$, it will receive a mark $1$ and $0$ otherwise.  Then all the marks are then added and divided by $n$, the number of points in $\Phi$.  This is an estimate of the probability that a point in the PP has its nearest neighbor at a distance less than $r$~\cite{Illian2008}.\par
%
Mathematically in the Palm sense, the mean and probabilities are show in equations~\eqref{eq:mean} and \eqref{eq:prob} respectively.
%
\begin{equation}\label{eq:mean}
  \textbf{E}_\textbf{0}(\Phi(b(\textbf{0},r) \setminus \{\textbf{0}\}))
\end{equation}
\begin{equation}\label{eq:prob}
  \textbf{P}_\textbf{0}(\Phi(b(\textbf{0},r) \setminus \{\textbf{0}\})>0)
\end{equation}
%
\section{Summary of Existing Work}
%
\cite{baccelli2009stochasticVII}
\begin{itemize}
  \item For CSMA: provides probability of transmission (media access), coverage probabilities, and rough estimates for successful transmission density
  \item For Aloha: provides
\end{itemize}

\section{CSMA}
%
Representation of a CSMA network requires use of the Mat\'ern hard-core PP.  This PP requires a minimum distance $r$ between points, or in our case transmitting nodes.  A Mat\'ern PP is generated from a marked PPP, where each point $x$ uniformly selects a mark $m(x)\in[0,1]$.  Points with marks larger than their neighboring point's marks are deleted if their positions are within range $r$.  Mathematically if $m(x)<m(y)$ and $\|x-y\| \leq r$, point $y$ is deleted.  This is process called conservative, since points that are already deleted can delete other points with larger marks. This is inaccurate in CSMA networks, but it is a common technique in the literature literature~\cite{Nguyen2007}.\par
%
To analyze the intensity of such PP, requires conditioning on existence of points at specific locations.  Palm theory formalizes the notion of the conditional distribution of a PP given it has a point at some location. Note that for a PP without a fixed \textit{atom} at this particular location, the probability of the condition is equal to zero and the basic discrete definition of the conditional probability does not apply~\cite{baccelli2009stochastic}.\par
%
\subsection{Slotted Aloha}
%
Before we considered the complicated case of CSMA, let us first look at a simpler case of slotted aloha with PPP distributed transmitters.  In this case we will define the coverage, or non-outage, probability for a \textit{typical} node.  A transmit receive pair is able to communicate if the SINR at the receiver is above some threshold $T$.  To say it in another way, transmitter $X_i$ \textit{covers} receiver $y_i$ in the reference time slot if equation~\eqref{eq:sinr} is satisfied.
%
\begin{equation}\label{eq:sinr}
  SINR_i = \frac{F_i^i/l(\|X_i - y_i\|)}{W + I_i^1} \geq T
\end{equation}
%
Where $I_i^1 = \sum_{X_j \in \hat{\Phi}^1,j \neq i} F_j^i/l(\|X_j-y_i\|) $ is the shot noise associated with the marked PP $\hat{\Phi}^1$. Now under this condition let us apply the mark $\delta_i$ to each transmitter $x_i$ meeting this condition.  The newly marked points form the PP $\hat{\Phi}$.
%
\todo[size=\small, color=green!40]{More introduction needed on next step to explain more on ideas behind using Palm Distributions}%
\par
%
We denote $\textbf{P}^0$ as the Palm probability associated to the $\delta$ marked stationary PP $\hat{\Phi}$.  By probability of coverage of a typical node given it is a transmitter, we show is equation~\eqref{eq:coverage}
%
\begin{equation}\label{eq:coverage}
  \begin{split}
  \textbf{P}^0\{\delta_0=1|e_0=1)\} &= \textbf{E}^0[\delta_0|e_0=1] \\
  &= \frac{1}{\lambda_1\|B\|}\textbf{E}[\sum_i \delta_i \mathbb{1}(x_i \in B)]
\end{split}
\end{equation}
%
Equation~\eqref{eq:coverage} line one is understood as given the central point \textbf{0} is selected/able to transmit ($e_1=1$), the Palm Probability that the receiver $y_0$ has significant SINR (equation~\eqref{eq:coverage} is met) from transmitter $x_0$ located at \textbf{0}.  This probability is equal to the average $\delta_0$ mark of the central located points of transmitters which are selected to transmit, seen by the r.h.s. of~\eqref{eq:coverage}.\par
%
Unlike in Aloha type MAC schemes, in this Mat\'ern CSMA model, the probability of medium access of a typical node $p = \textbf{E}^0[e_i]$ is not given \textit{a priori} and it has to be determined.  $e_i$ are simply the medium access indicators, marking the points that survived the Mat\'ern process creation from the original PPP $\Phi$.  When calculating the probability of access to the channel, we must first calculate the expected number of neighbors of a typical node, which we denote as $\bar{N}(\lambda)$.  From~\cite{baccelli2009stochasticVII} we have Equation~\eqref{eq:neighbors}.
%
\begin{equation}\label{eq:neighbors}
  \begin{split}
  \bar{N} &= \textbf{E}^0[\sum_{(x_j,m_i,\textbf{F}_j^0)\in\hat{\Phi}}\mathbb{1}(\frac{F_j^0}{l(\|x_i-x_j\|)}\geq P_0)] \\
  &= \lambda\int_{\mathbb{R}^2} \textbf{P}\{F\geq P_0l(\|x\|) \}dx \\
  &= 2\pi \lambda \int_0^\infty (1-G(P_0l(r)))rdr
\end{split}
\end{equation}
%
In equation~\eqref{eq:neighbors} $F_j^0$ denotes the power emitted by node $j$, given that it is selected to transmit, towards the origin $\textbf{0}$.  $l()$ is the pathloss model used in the scenario (Ex: Power Law $l(r)=r^{-\alpha}$),  $P_0$ is the detection threshold for the receiver at the $\textbf{0}$.  $G$ is simply the CDF of the fading of the transmitters.  If $F$ is Rayleigh fading (exponential $F$ with mean $\frac{1}{\mu}$), $\bar{N}$ becomes:\par
%
\begin{equation}
  \bar{N} = 2 \pi \lambda \int_0^\infty  e^{-P_0 \mu l(r)} r dr
\end{equation}
%
Taking equation~\eqref{eq:neighbors} and rewriting the probability of medium access, we get equation~\eqref{eq:probra} based on void probability.
%
\begin{equation}\label{eq:probra}
  p = \textbf{E}[e_0] = \int e^{-\bar{N}t} dt= \frac{1-e^{\bar{N}}}{\bar{N}}
\end{equation}
%
Note by the ergodicity of the model that $p$ represents some spatial average, namely an average over all nodes of the PPP realizations. More precisely $\lambda p$ is the density of the Mat\'ern PP of nodes authorized to transmit at a given time slot. However, it does not mean that any given
node is selected by the MAC with the time frequency $p$, even if in different time slots the marks of each point are re-sampled independently.  This $p$ is better understood as a network wide proportion, rather than a local average.



\section{System Model}
x

\section{Notation}
%
\begin{tabular}{ l | p{10cm} }
  $e_i$ & the medium access indicator of node i; ($e_i = 1$ if node i is allowed to transmit in the considered time slot and $0$ otherwise). The random variables $e_i$ are hence \textit{i.i.d.} and independent of everything else, with $\textbf{P}](e_i = 1) = p$ ($p$ is the MAP). \\
  \hline
  $F_i^j$ & denotes the virtual power emitted by node $i$ (provided $e_i = 1$) towards receiver $y_j$ \\
  \hline
\end{tabular}


%
\bibliographystyle{IEEEtran}\bibliography{IEEEabrv,refs.bib}
%

\end{document}
